\documentclass[../fstates.tex,../../main.tex]{subfiles}

\begin{document}
\section{Excited States via the Bethe Ansatz}

The Bethe ansatz is an exact method for the calculations of the eigenvalues and eigenvectors of a select class of quantum many-body systems. It was presented in 1931 by Hans Bethe\cite{bethe} to obtain the exact eigenvalues and eigenvectors of the one-dimensional spin-\sfrac{1}{2} Heisenberg model, which, as we have already seen, is a linear array of electrons with uniform exchange interaction between nearest neighbours. Although nowdays many other quantum many-body systems are known to be solvable by a variant of the Bethe ansatz, we will present the original work, already notoriously complicated, solving the 1D ferromagnetic spin-\sfrac{1}{2} Heisenberg model, with a modern approach.\\

The idea behind the Bethe ansatz is to consider the chosen ground state, which is an eigenstate of the Hamiltonian, with all the spins up and then flipping some spins.\\

We choose the ground state to be
\begin{equation}
    \ket{0}=\ket{\uparrow\uparrow\uparrow\ldots\uparrow}
\end{equation}
with ground energy
\begin{equation}
    H\ket{0}=E_{0}=-\frac{JN}{4}
\end{equation}

The next step is to flip a spin, which can be done using the spin ladder operators ($S^{\pm}$). A state with $M$ flipped spins is written as
\begin{equation}
    \ket{n_{1},\dots, n_{M}}=S_{n_{1}}^{-}\dots S_{M}^{-}\ket{0}
\end{equation}

So the eigenstate is of the form
\begin{equation}
    \ket{\psi}=\sum_{1\le n_{1}\le\dots\le n_{M}\le N} a(n_{1},\dots,n_{M})\ket{n_{1},\dots, n_{M}}
\end{equation}
where $a(n_{1},\dots,n_{M})$ represent unknown coefficients. Because of the periodicity of the lattice, $a(n_{1}+N,\dots,n_{M})=a(n_{1},\dots,n_{M})$, the Bethe ansatz proposes that the form of those coefficients is
\begin{equation}
    a(n_{1},\dots,n_{M})=\sum_{\sigma\in S_{M}} A_{\sigma}e^{ip_{\sigma_{i}}n_{i}}
\end{equation}
which is a plain-wave ansatz. Each such flip\footnote{Up to degeneration.} creates excitations that behave like quasi-particles called magnons, equivalently known and as quantized spin waves in the wave picture of quantum mechanics. 

\section{A Single Magnon}
Here we will only discuss the case for a single magnon, cases with more magnons are extensivelly discussed in \cite{fazekas, deleeuw,plantz}\\

If we flip a single spin, there are $N$ different states possible. If we call the lattice site with the flipped spin $n$, then the state will be
\begin{equation}
    \ket{n}:=S^{-}_{n}\ket{0}=\ket{\ldots\downarrow_{n}\ldots}
\end{equation}
and the Hamiltonian acting on it will give us
\begin{equation}
    H\ket{n}=-J\left(\frac{1}{2}\ket{n-1}+\frac{1}{2}\ket{n+1}+\ket{n}\right)
\end{equation}

We can see here that the flipped spin behaves like some sort of quasi-particle, being able to travel around sites or stay put.\\

If we now want to study this new quasi-particle the most natural way to start is to write down the eigenvector of the momentum $p$
\begin{equation}
    \ket{p}=\sum_{n}e^{ipn}\ket{n}
\end{equation}
which is the discrete version of a plane-wave that we got by taking $a(n)=e^{ipn}$.\\

If we now go back and look at the Hamiltonian action on a flipped spin at some position $n$, we can see that it has a contribution from its neighbouring sites
\begin{equation}
    H\ket{p}=\ldots-\frac{J}{2}\left[e^{ip(n+1)}+e^{ip(n-1)}-e^{ipn}\right]\ket{n}\ldots
\end{equation}

Therefore we see that $\ket{p}$ is an eigenstate of the Hamiltonian, with corresponding energy
\begin{equation}
    E=2J\sin^{2}\left(\frac{p}{2}\right)
\end{equation}
so the magnon is a quasi-particle with momentum $p$ and energy $E$ moving in the ground state.\\

As we impose periodicity, it acts as a quantization condition on the momentum which can be seen by acting with the Hamiltonian on the $N$th site. Since we get a contribution from $n_{1}$ instead of $n_{N+1}$, we find that this is an eigenstate with eigenvalue E
\begin{equation}
    e^{ipN}=1
\end{equation}
which is the momentum quantization condition for a particle on a circle of length $N$. 

\end{document}