\documentclass[../rsrg.tex,../../main.tex]{subfiles}

\begin{document}

The idea behind RG processes came when, in 1966, Leo Kadanoff\cite{kadanoff} proposed that spins can be transformed into superspins, shaped from blocks of spins, for Ising-like models and proving some empirical\footnote{Empirical at the time.} relations.\\

% In 1971, Kenneth Wilson published two papers using Kadanoff's idea to shape this transformation into a useful tool and in 1974 he used this tool to solve the Kondo problem \cite{wilson}, which dealt with the effect of a magnetic impurity on conduction band electrons of a metal. His solution used the division of the entire lattice into shells, based around the impurity, so that the furthest the shell was from the impurity, the nearer to the Fermi surface the electrons were supposed to be. Integrating the shells iteratively, from the center outwards, the external electrons were only able to see the impurity as sheltered by the inner shells.\\

Phase transitions had mostly been studied in classical systems, where the Ising model has been solved exactly in two dimensions\cite{onsager} and the Kondo problem was famously solved by Wilson in 1975\cite{wilson}. If we want to study the low temperature behaviour of physical systems, we have to take into account the quantum nature of such systems, since it affects phase transition phenomena.\\

This is because the universality ideas that have emerged lead to the fact that the critical behaviour should be affected by quantum effects at low temperatures, which implies that the existence of quantum transitions at $T=0$ creates quantum-classical crossover phenomena in classical low temperature transitions. Because of this, real-space renormalisation group methods have been expanded to quantum systems at $T\ne0$, whilst at $T=0$ the block renormalisation group method has been introduced to study the ground state and the excited states of many-body quantum systems, allowing for the study of transitions which take place in the ground state of the system.\\

In 1977, H.P. van der Braak et al\cite{braak}, used the Block RG method to study the AF Heisenberg model while some years later, in 1979, Dasgupta, Ma and Hu\cite{dasgupta-hu, ma}, developed their own real space RG process to study the low-temperature properties of the random AF Heisenberg model in 1D. 

\end{document}
