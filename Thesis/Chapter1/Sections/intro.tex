\documentclass[../intro.tex,../../main.tex]{subfiles}
\begin{document}

A quantum spin chain consists of a one-dimensional lattice with $N$ sites, where on each site we consider a spin particle, in the case of a S-\sfrac{1}{2} particle an electron. This electron can either have spin up (denoted by $\ket{\uparrow}$) or down (denoted by $\ket{\downarrow}$) and, therefore, any electron exists in a linear state $a\ket{\uparrow} + b\ket{\downarrow}$, which generates a local two-dimensional Hilbert space. Since the lattice is of size $N$, we also have $N$ electrons, so the total Hilbert space in which the states live is
\begin{equation}
    H=\bigotimes_{N}\mathbb{C}^{2}
\end{equation}

The spin operators $S_{i}^{x,y,z}$ act on each site $i$ and satisfy the local commutation relations
\begin{equation}
    \comm{S_{i}^{a}}{S_{j}^{b}}=\delta_{ij}\epsilon^{abc}S_{i}^{c},\ i\ne j
\end{equation}

The Hamiltonian describes a nearest neighbour interaction between the spins,
\begin{equation}
    H=-J\sum_{i}^{N}\vec{S}_{i}\vec{S}_{i+1}
\end{equation}
and we also demand that $\vec{S}_{N+1}=\vec{S}_{1}$.\\

If we expand the Hamiltonian writing the vector components, it is a special case of a more general Hamiltonian, with the form
\begin{equation}
    H=-\sum_{i=1}^{L} (J^{x}S_{i}^{x}S_{i+1}^{x} +J^{y}S_{i}^{y}S_{i+1}^{y} +J^{z}S_{i}^{z}S_{i+1}^{z})
\end{equation}
Depending on the nature of $J^{x},\ J^{y},\ J^{z}$, one can find five different models, called
\begin{itemize}
    \item The $XYZ-model$, where $J^{x}\ne J^{y}\ne J^{z}$
    \item The $XXZ-model$, where $J^{x}=J^{y}\ne J^{z}$
    \item The $XXX-model$, where $J^{x}=J^{y}=J^{z}$
    \item The $XY-model$, where $J^{x}\ne J^{y}$, and $J^{z}=0$
    \item The $XX-model$, where $J^{x}=J^{y}$, and $J^{z}=0$
\end{itemize}
Without loss of generality, we can assume that the signs of the $J^{x}$ are the same for the whole chain, as well as for the $y$ and $z$ components individually. By rotating the spins around the $z$-axis, we can always assume that $J^{x}>0$ and $J^{y}<0$, so that the ferromagnetical nature of the model is characterised solely by the $J^{z}$. $J^{z}<0$ means the antiferromagnietic, while $J^{z}>0$ means the ferromagnetic model.\\

Another way to write the Hamiltonian, which will be of use for the later Sections, is via the spin ladder operators ($S^{\pm}=S^{x}\pm iS^{y}$), which have the properties that
\begin{equation}
    \begin{array}{cc}
        S^{+}\ket{\uparrow}=0, & S^{+}\ket{\downarrow}=\ket{\uparrow} \\[10pt]
        S^{-}\ket{\uparrow}=\ket{\downarrow}, & S^{-}\ket{\downarrow}=0
    \end{array}
\end{equation}

The Hamiltonian then becomes
\begin{equation}
    H=J\sum_{i}^{N}\left[\frac{1}{2}(S_{i}^{+}S_{i+1}^{-}+S_{i}^{-}S_{i+1}^{+})\right]+J^{z}\sum_{i}^{N}(S_{i}^{z}S_{i+1}^{z})
\end{equation}

We will use this form implicitly while dealing with the Bethe Ansatz and explicitly when we deal with the Jordan-Wigner transformation.\\

By looking at the symmetries of the system we can reduce the effective size of the Hamiltonian. Consider the operator
\begin{equation}
    S^{z}=\sum_{i}^{N}S_{i}^{z}
\end{equation}
which measures the total number or up or down spins. Since it commutes with the Hamiltonian, we can restrict the subsets of a fixed number of spins up or down. As we extend this to all spin operators and define
\begin{equation}
    \vec{S}=\sum_{i}^{N}\vec{S}_{i}
\end{equation}
we see that this also commutes with the Hamiltonian. Since the spin operators form an SU(2) algebra, the spin chain also has SU(2) as a symmetry algebra, meaning that it is symmetric under global rotations of the unit sphere in which the spins exist. This also implies that the eigenstates of the Hamiltonian will arrange themselves in multiplets with respect to the algebra of SU(2).

\end{document}