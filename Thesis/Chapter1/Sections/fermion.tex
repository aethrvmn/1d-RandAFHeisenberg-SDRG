\documentclass[../intro.tex,../../main.tex]{subfiles}
\begin{document}

\section{Relation to the Free Fermion Model}

Jordan and Wigner\cite{jorwig} observed, in 1928, that the up and down states of a single spin can be seen as an occupied or empty fermion state, enabling them to make the mapping
\begin{equation}
    \begin{array}{cc}
        \ket{\uparrow}=f^{\dagger}\ket{0}, & \ket{\downarrow}=\ket{0} 
    \end{array}
\end{equation}
allowing an explicit representation of the spin ladder operators
\begin{equation}
    \begin{split}
        S^{+} = f^{\dagger} &=\left[\begin{array}{cc}
            0 & 1 \\
            0 & 0
        \end{array}\right]\\
        S^{-} = f &=\left[\begin{array}{cc}
            0 & 0 \\
            1 & 0
        \end{array}\right]\\
    \end{split}
\end{equation}
while the $z$ component of the spin operator is written as
\begin{equation}
    S^{z}=\frac{1}{2}\Big[\ketbra{\uparrow}-\ketbra{\downarrow}\Big]=f^{\dagger}f-\frac{1}{2}
\end{equation}

The $x \text{ and } y$ components are given by
\begin{equation}
\begin{split}
    S^{x}&=\frac{1}{2}(f^{\dagger}+f)\\
    S^{y}&=\frac{1}{2i}(f^{\dagger}-f)
\end{split}
\end{equation}

If there are more than one spin particles, the representation needs to be modified, due to the fact that while independent spin operators commute, independent fermions anticommute. We can overcome this difficulty by attaching a phase factor, called a string, to the fermions, so that in the Jordan-Wigner representation, the spin operator at site $j$ is
\begin{equation}
    S_{j}^{+}=f_{j}^{\dagger}e^{i\phi_{j}}
\end{equation}
where the phase operator $\phi_{j}$ contains the sum of all fermion occupancies at sites left of $j$
\begin{equation}
    \phi_{j}=\pi\sum_{l<j}n_{l}
\end{equation}
The operator $e^{i\phi_{j}}$ is thus known as a string operator.\\

Therefore, the complete transformation is
\begin{equation}
    \begin{split}
        S^{+}_{j} &= f_{j}^{\dagger}e^{i\pi\sum_{l<j}n_{l}}\\
        S^{-}_{j} &= f_{j}e^{-i\pi\sum_{l<j}n_{l}}\\
        S^{z}_{j} &= f_{j}^{\dagger}f_{j}-\frac{1}{2}
    \end{split}
\end{equation}

In layman's terms we took a spin and transformed it into a fermion interacting with a string
\begin{equation*}
    \text{spin}\leftrightarrow\text{fermion}\times\text{string}
\end{equation*}

If we now look at the Heisenberg Hamiltonian in the spin ladder operators form
\begin{equation}
H=J\sum_{i}^{N}\left[\frac{1}{2}(S_{i}^{+}S_{i+1}^{-}+S_{i}^{-}S_{i+1}^{+})\right]+J^{z}\sum_{i}^{N}(S_{i}^{z}S_{i+1}^{z})
\end{equation}
the way to produce the fermionic form is to notice that, in the first term, all terms in the string operators cancel except for $e^{i\pi n_{j}}$, which doesn't effect the Hamiltonian and is therefore ignored. The first part of the first term becomes
\begin{equation}
    \frac{J}{2}\sum_{i}^{N}S_{i}^{+}S_{i+1}^{-}=\frac{J}{2}\sum_{j}f_{j}^{\dagger}f_{j+1}
\end{equation}
and the second, being the Hermitian conjugate of the first, becomes
\begin{equation}
    \frac{J}{2}\sum_{i}^{N}S_{i}^{-}S_{i+1}^{+}=\frac{J}{2}\sum_{j}f_{j}f_{j+1}^{\dagger}
\end{equation}

We see that this term, being the transverse component of the interaction, includes a hopping term in the fermionized Hamiltonian. Of note is the fact that should the spin interaction include next-nearest neighbours, the string terms would make a comeback.\\

The $z$ component of the Hamiltonian becomes
\begin{equation}
    J^{z}\sum_{i}^{N}(S_{i}^{z}S_{i+1}^{z})=J^{z}\sum_{j}\left(n_{j+1}-\frac{1}{2}\right)\left(n_{j}-\frac{1}{2}\right)
\end{equation}

So that the complete transformed Hamiltonian is
\begin{equation}
    H=-\frac{J}{2}\sum_{j}(f_{j}^{\dagger}f_{j+1}+f_{j}f_{j+1}^{\dagger})+J^{z}\sum_{j}n_{j}-J^{z}\sum_{j}n_{j}n_{j+1}
\end{equation}

There are two important things to note. The first one is that the ferromagnetic interactions mean that the spin fermions are actually attracted to one another. The second one is that the $XY$ model ($J^{z}=0$) has no interaction term, so this can be mapped to a free fermion model. The latter was first shown by Lieb et al\cite{lieb} who used this equivalence to study the physical properties of the $XY$ model.
\end{document}