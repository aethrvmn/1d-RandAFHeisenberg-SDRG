\documentclass[../main.tex]{subfiles}

\begin{document}
Η μέθοδος επανακανονικοποίησης ισχυρής αταξίας, όπως παρουσιάστηκε πρώτα από τους Dasgupta, Ma και Hu, δίνει ασύμπτωτικά ακριβή αποτελέσματα σε κατανομές όπου η αταξία αυξάνεται χωρίς όριο σε μεγάλες κλίμακες, ενώ αργότερα, ο Fisher, ο οποίος επέκτεινε την ιδέα, υπολόγισε και οριακές τιμές καθώς και συντελεστές κλίμακας για τυχαίες αλυσίδες σπιν.\\

Αυτά τα αποτελέσματα δεν ήταν παρά τα πρώτα πολλών που ακολούθησαν μέσω εντατικής έρευνας, πρώτα σε τυχαία κβαντικά συστήματα ενώ ύστερα προέκυψε και η επέκτασή τους σε κλασσικά άτακτα μοντέλα. Οι προηγούμενες μεθόδοι επανακανονικοποίησης αντιμετώπιζαν το σύστημα σε έναν ομογενή χώρο, το οποίο επέτρεπε την ομαδοποίηση των σπιν σε υπερσπίν και ενώ σε μη-τυχαία συστήματα αυτή η ομογένεια είναι φυσικά επιβεβαιώσημη, τήθεται υπό ερώτημα παρουσία άτακτων συστημάτων. Η μέθοδος επανακανονικοποίησης ισχυρής αταξίας αντιθέτως, επανακανονικοποιεί τον χώρο με μη-ομογενή τρόπο, επιτρέποντας έτσι τον καλλίτερο χειρισμό τοπικών αταξιών.\\

Πιο συγκεκριμένα, η αλυσίδα XX, που πρώτα μελέτησε ο Fisher, δίνει ακριβή αποτελέσματα για την συμπεριφορά φάσεων στις οποίες κυριαρχεί η τυχαιότητα, καθώς και για την κρίσιμη συμπεριφορά κοντά στις διάφορες αλλαγές φάσης που προκύπτουν σε μηδενική θερμοκρασία. Μελετώντας τις ιδιότητες της αντιφερρομαγνητικής αλυσίδας με σπιν-\sfrac{1}{2} με τυχαίους δεσμούς, αναλύουμε της συμπεριφορά χαμηλής ενέργειας, σπώντας τον ισχυρότερο δεσμό και αντικαθιστώντας τον με έναν ενεργό δεσμό μεταξύ πλησιέστερων γειτόνων. Επαναλαμβάνοντας την διαδικασία, η κατανομή φαρδαίνει, βελτιώνοντας την ακρίβεια της προσέγγισης.\\

Η δομή της εργασίας είναι η εξής. Πρώτα εισαγάγουμε το πρότυπο Heisenberg και δείχνουμε την συσχέτισή του με τα πρότυπα του Ising και των ελευθέρων φερμιονίων, το λύνουμε ακριβώς στην φερρομαγνητική περίπτωση μέσω του άνσατζ του Bethe, και εισαγάγουμε την μέθοδο επανακανονικοποίησης μέσω blocks για την αντιφερρομαγνητική περίπτωση. Στην συνέχεια παρουσιάζουμε την μέθοδο επανακανονικοποίησης ισχυρής αταξίας, χρησιμοποιώντας μια σύγχρονη μορφή της μεθόδου που ο Fisher χρησιμοποίησε για να λύσει την τυχαία αντιφερρομαγνητική XX αλυσίδα. Τέλος, παρουσιάζουμε τις μεθόδους που φτιάξαμε για να μοντελοποιήσουμε την διαδικασία με χρήση της προγραμματιστική γλώσσας Python.
\end{document}