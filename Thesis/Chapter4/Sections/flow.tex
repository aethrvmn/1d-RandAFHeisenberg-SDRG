\documentclass[../sdrg,../../main.tex]{subfiles}
\begin{document}
\section{RG Flow Equation for the XX Chain}

For the rest of the discussion we shall only consider the simplest case, that of the $XX$ chain with $J^{x}=J^{y}=J$. The recursion relation then becomes
\begin{equation}
\label{eq:recursionrel2}
    J^{z} = 0,\ \tilde{J}= \frac{J_{1}J_{3}}{J_{2}}
\end{equation}
and defining the strongest bond is simply $\Omega := \max(J)$.\\

Because of the new recursion relation, it is convenient to transform our variables to logarithmic, defining
\begin{equation}
\begin{split}
    \Gamma &:= -\ln(\Omega)\\
    \zeta &:= \ln(\flatfrac{\Omega}{J})
\end{split}
\end{equation}
where obviously $\zeta \ge 0$ and large $\zeta \rightarrow \text{ small } J$.\\

Again the recursion relation changes. This time \eqref{eq:recursionrel2} becomes
\begin{equation}
    \zeta = \zeta_{1}+\zeta_{3}-\zeta_{\Omega} = \zeta_{1}+\zeta_{3}=\zeta_{-}+\zeta_{+}
\end{equation}

Having defined these new logarithmic variables, we can express the distribution of bonds as $P(\zeta,\Gamma)$, with the probability of a bond $\zeta$ at a fixed scale $\Gamma$ being $P(\zeta,\Gamma)\dd \zeta=\dd P(\zeta,\Gamma)$. With every step of the elimination transformation, $\Gamma$ changes to $\Gamma + \delta\Gamma$, firstly because of the change of $\zeta \to \zeta'$
\begin{equation}
    \zeta' = \ln(\frac{\Omega'}{J}) = \ln(\frac{\Omega}{J}) - \Gamma + \Gamma' = \zeta + \delta\Gamma
\end{equation}
and secondly from the fact that in the elimination process, a new bond is added, namely $\tilde{J}$.\\

Combining those two contributions, near the limit where $\delta\Gamma\to 0$, the distribution $P(\zeta,\Gamma)\dd\zeta$ of bonds $\zeta$ at scale $\Gamma$ becomes
\begin{equation}
    \pdv{\Gamma}P(\zeta,\Gamma) = \pdv{\zeta}P(\zeta,\Gamma) + P(0,\Gamma)\int_{0}^{\infty}\dd\zeta_{-}\int_{0}^{\infty}\dd\zeta_{+}\delta(\zeta-\zeta_{+}-\zeta_{-})P(\zeta_{+},\Gamma)P(\zeta_{-},\Gamma)
\end{equation}
where $P(0,\Gamma)\dd\Gamma$ is the fraction of bonds with $\zeta$ in the range $0\text{ to } \dd\Gamma$. Since $\zeta_{-}, \zeta_{+}$ represent the bonds on each side (left and right respectively), the recursion relation makes an appearance inside the delta function of the integral, which ensures that for each $\zeta$, we are adding the corresponding probability of obtaining an effective bond with that value to the new distribution.
\end{document}