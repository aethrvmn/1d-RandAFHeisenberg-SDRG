\documentclass[../sdrg,../../main.tex]{subfiles}
\begin{document}
Renormalization methods in pure systems usually involve a finite number of coupling constants, in contrast with renormalization methods in disordered systems, which involve probability distributions that live in an infinite dimensional space. This increases the difficulty of the study of the RG flow, since now we study complex functionals instead of critical exponents and the fixed points are much harder to find. This extra difficulty usually leads to the necessity of numerical solutions or to additional approximations consisting of projections into finite spaces by choosing certain analytical forms for distributions with a limited number of parameters. There is a small number of RG flows that are simple enough to be analyzed completely, whose fixed point distributions usually have interesting probabilistic interpretations, the Dasgupta-Ma RG method being one of them.\\

The Dasgupta-Ma RG method, as introduced by Dasgupta and Ma\cite{ma}, has two important properties. The first one is that the renormalization concerns the extreme value of a random variable, which determines the scale and evolves via the renormalization, and serves as the cut-off point of the renormalized distribution. The second property is that the renormalization is local in space, meaning that in each step only the immediate neighbours of the aforementioned random variable are concerned by the RG procedure.
\end{document}